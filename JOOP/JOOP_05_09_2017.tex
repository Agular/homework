\title{JOOP}
\author{
        Ragnar Luga \\
                Infotehnoloogia teaduskond\\
                Informaatika\\
}
\date{\today}

\documentclass[12pt]{article}
\usepackage[utf8]{inputenc}
\usepackage{listings}

\begin{document}
\begin{titlepage}
\maketitle
\end{titlepage}


\section{Sissejuhatus}

\begin{itemize}
\item 8. nädalal (23.-29. oktoober) loengut ja praktikume ei toimu.
\item [\textbf{Loengutestid}]
\item 7. nädalal ja 16. nädalal loengu lõppus.
\item [\textbf{Kontrolltööd}]
\item 7. nädalal ja 15. nädalal.
\end{itemize}

\begin{enumerate}
\item [\textbf{Põhiteemad}]
\item Objektid ja disain.
\item Realisatsiooni varjamine, API disain.
\item Andmetüüpide disain, vood.
\item Polümorfism, ülelaadimine, ülekirjutamine.
\item Java Virtuaalmasina (JVM) tööpõhimõtted.
\item Liidesed, erindid. Tüübituvastus.
\item Kapseldamine. Kompositsioon.
\item Töö lõimedes (threads), paralleeltöö.
\item Predikaadid, lambdad. Spring Boot.
\end{enumerate}

\section{Java ja JVM}

\begin{itemize}
\item [\textbf{Tüpiseerimine}]
\item Static typing
\item Dynamic typing
\item Duck typing
\item JRE - käivitamiseks
\item JDK - koodi kijutamiseks

\end{itemize}

\section{Pakett}

\begin{itemize}
\item Viis koodi struktureerimiseks
\end{itemize}


\end{document}